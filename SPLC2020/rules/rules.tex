%%
%% This is file `sample-sigconf.tex',
%% generated with the docstrip utility.
%%
%% The original source files were:
%%
%% samples.dtx  (with options: `sigconf')
%%
%% IMPORTANT NOTICE:
%%
%% For the copyright see the source file.
%%
%% Any modified versions of this file must be renamed
%% with new filenames distinct from sample-sigconf.tex.
%%
%% For distribution of the original source see the terms
%% for copying and modification in the file samples.dtx.
%%
%% This generated file may be distributed as long as the
%% original source files, as listed above, are part of the
%% same distribution. (The sources need not necessarily be
%% in the same archive or directory.)
%%
%% The first command in your LaTeX source must be the \documentclass command.
\documentclass[sigconf]{acmart}

%%
%% \BibTeX command to typeset BibTeX logo in the docs
\AtBeginDocument{%
  \providecommand\BibTeX{{%
    \normalfont B\kern-0.5em{\scshape i\kern-0.25em b}\kern-0.8em\TeX}}}

%% Rights management information.  This information is sent to you
%% when you complete the rights form.  These commands have SAMPLE
%% values in them; it is your responsibility as an author to replace
%% the commands and values with those provided to you when you
%% complete the rights form.
\setcopyright{acmcopyright}
\copyrightyear{2018}
\acmYear{2018}
\acmDOI{10.1145/1122445.1122456}

%% These commands are for a PROCEEDINGS abstract or paper.
\acmConference[SPLC'20]{24th ACM International Systems and Software Product
Line Conference}{19--23 October, 2020}{Montreal, Canada}
% \acmConference[Woodstock '18]{Woodstock '18: ACM Symposium on Neural
%   Gaze Detection}{June 03--05, 2018}{Woodstock, NY}
% \acmBooktitle{Woodstock '18: ACM Symposium on Neural Gaze Detection,
%   June 03--05, 2018, Woodstock, NY}
% \acmPrice{15.00}
% \acmISBN{978-1-4503-XXXX-X/18/06}


%%
%% Submission ID.
%% Use this when submitting an article to a sponsored event. You'll
%% receive a unique submission ID from the organizers
%% of the event, and this ID should be used as the parameter to this command.
%%\acmSubmissionID{123-A56-BU3}

%%
%% The majority of ACM publications use numbered citations and
%% references.  The command \citestyle{authoryear} switches to the
%% "author year" style.
%%
%% If you are preparing content for an event
%% sponsored by ACM SIGGRAPH, you must use the "author year" style of
%% citations and references.
%% Uncommenting
%% the next command will enable that style.
%%\citestyle{acmauthoryear}

%%
%% end of the preamble, start of the body of the document source.

%%%%%%%%%%%%%%%%%%%%%%%%%%%%%%%%%%%%%%%%%%%%%%%%%%%%%%%%%%%%%%%%%%%%%%%%%%%%%%%%
%%%%%%%%%%%%%%%%%%%%%%%%%%%%%%% Packages and Commands %%%%%%%%%%%%%%%%%%%%%%%%%%
%% Niceties
\usepackage{booktabs}   %% For formal tables:
                        %% http://ctan.org/pkg/booktabs
\usepackage{subcaption} %% For complex figures with subfigures/subcaptions
                        %% http://ctan.org/pkg/subcaption
\usepackage{caption}
\usepackage{listings}   %% for code snippets in
\usepackage{enumitem,kantlipsum}
\usepackage{amsmath}
\usepackage{tikz}       %% assertion stack visualization
\usetikzlibrary{matrix}
\usetikzlibrary{arrows}
\usetikzlibrary{positioning}
\usepackage{xcolor}     %% for coloring blocks of text to show sharing

%%%% Commands used throughout the paper
\usepackage{paperCommands}
\usepackage{lambda}
\usepackage{cc}
% \usepackage{vlc}

%%%% Packages for Implementation
% \usepackage{mathtools}                  %% math symbols in syntax
\usepackage{mathpartir} %%% for inference rules
\usepackage{amsthm}

%%%% Commands for Implementation
\newcommand{\unit}{\ensuremath{\bullet}}                           % unit in math mode
\newcommand{\featMF}[1]{\ensuremath{\kernfix{fmf_{#1}}}}         % fmf macro w/ subscript
\newcommand{\fmf}{\featMF{}}                   % feature model no subscript
\newcommand{\fmfT}{\featMF{True}}                    % fmf with true subscript
\newcommand{\fmfF}{\featMF{False}}
\newcommand{\mdl}[1]{$\kernfix{mdl_{#1}}$}           % model macro
\newcommand{\mdls}{$mdls$}                           % mdls helper
\newcommand{\SatVar}{\ensuremath{\_Sat}}                        %_Sat
\newcommand{\SatModel}{\mdl{Sat}}
\newcommand{\Satfmf}{\featMF{_{\_Sat}}}

%% bool values for model figures
\newcommand{\fls}{\texttt{F}}
\newcommand{\tru}{\texttt{T}}

\newcommand{\dimd}{$d$}
\newcommand{\singleton}[1]{\{#1\}}

%%%% Vsolve SPL section commands
%% Simple macros for Linux SPL example
\newcommand{\Lx}[1]{\ensuremath{L_{#1}}}
\newcommand{\FM}[1]{\ensuremath{\kernfix{FM}_{#1}}}

\newcommand{\LNot}{\Lx{0}}
\newcommand{\FMNot}{\FM{0}}

\newcommand{\LOne}{\Lx{1}}
\newcommand{\FMOne}{\FM{1}}

\newcommand{\LTwo}{\Lx{2}}
\newcommand{\FMTwo}{\FM{2}}

% features in Linux kernel
\newcommand{\Fmitigate}{\kernfix{mitigations}}
\newcommand{\Fspectre}{\kernfix{spectre\_v2}}
\newcommand{\Fnospec}{\kernfix{nospec\_store\_bypass\text{-}disable}}
\newcommand{\Flone}{\kernfix{l1tf}}
\newcommand{\Fpti}{\kernfix{pti}}

%%%% Experimental Methodology packages

\usepackage{multirow}
%%%%%% Lstlisting style
\lstset{
  frame=top,frame=bottom,
  basicstyle=\small\normalfont\sffamily,    % the size of the fonts that are used for the code
  stepnumber=1,                           % the step between two line-numbers. If it is 1 each line will be numbered
  numbersep=10pt,                         % how far the line-numbers are from the code
  tabsize=2,                              % tab size in blank spaces
  extendedchars=true,                     %
  breaklines=true,                        % sets automatic line breaking
  captionpos=t,                           % sets the caption-position to top
  mathescape=true,
  % stringstyle=\color{white}\ttfamily,
  % showspaces=false,
  % showtabs=false,
  % xleftmargin=17pt,
  % framexleftmargin=17pt,
  % framexrightmargin=17pt,
  % framexbottommargin=5pt,
  % framextopmargin=5pt,
  showstringspaces=false,
  escapeinside={(*}{*)},%
 }

 %%%%%%%%%%%%%%%%%%%%%%%%%%%%%%%% Results %%%%%%%%%%%%%%%%%%%%%%%%%%%%%%%%%%
\usepackage{array}

%%%%%%%%%%%%%%%%%%%%%%%%%%%%%%%% End Packages %%%%%%%%%%%%%%%%%%%%%%%%%%%%%%%%%%
%%%%%%%%%%%%%%%%%%%%%%%%%%%%%%%%%%%%%%%%%%%%%%%%%%%%%%%%%%%%%%%%%%%%%%%%%%%%%%%%





\begin{document}

%%%%%%%%%%%%%%%%%% Indentation settings in align mode
\raggedbottom
\setlength{\abovedisplayskip}{2.5pt}
\setlength{\belowdisplayskip}{2.5pt}
\setlength{\abovedisplayshortskip}{2.5pt}
\setlength{\belowdisplayshortskip}{2.5pt}


%%
%% The "title" command has an optional parameter,
%% allowing the author to define a "short title" to be used in page headers.
\title{Variational Satisfiability Solving Inference Rules}

%%
%% The "author" command and its associated commands are used to define
%% the authors and their affiliations.
%% Of note is the shared affiliation of the first two authors, and the
%% "authornote" and "authornotemark" commands
%% used to denote shared contribution to the research.

\author{Jeffrey M. Young}
% \authornote{Both authors contributed equally to this research.}
\email{youngjef@oregonstate.edu}
% \orcid{1234-5678-9012}
\affiliation{%
  \institution{Oregon State University}
  % \streetaddress{P.O. Box 1212}
  \city{Corvallis}
  \state{Oregon}
  \country{USA}
  % \postcode{43017-6221}
}
\author{Eric Walkingshaw}
% \authornotemark[1]
\email{walkiner@oregonstate.edu}
\affiliation{%
  \institution{Oregon State University}
  % \streetaddress{P.O. Box 1212}
  \city{Corvallis}
  \state{Oregon}
  \country{USA}
  % \postcode{43017-6221}
}

\author{Thomas Th{\"u}m}
\email{thomas.thuem@uni-ulm.de}
\affiliation{%
  \institution{University of Ulm}
  % \streetaddress{1 Th{\o}rv{\"a}ld Circle}
  \city{Ulm}
  \country{Germany}
}
%%
%% By default, the full list of authors will be used in the page
%% headers. Often, this list is too long, and will overlap
%% other information printed in the page headers. This command allows
%% the author to define a more concise list
%% of authors' names for this purpose.
\renewcommand{\shortauthors}{Young, Walkingshaw, Th{\"u}m}

%%
%% The abstract is a short summary of the work to be presented in the
%% article.

%%
%% The code below is generated by the tool at http://dl.acm.org/ccs.cfm.
%% Please copy and paste the code instead of the example below.
%%
%%
%% Keywords. The author(s) should pick words that accurately describe
%% the work being presented. Separate the keywords with commas.
% \keywords{satisfiability solving, feature modeling, software product lines, variation}

\maketitle

\begin{figure}
  \begin{mathpar}
  %%% Computation rules
  \inferrule*[Right=Ac-Gen]
  { r \notin \kernfix{dom}(\Delta) \\
    \texttt{spawn($\Delta, r$)} = (\Delta', s)}
  {(\Delta, r) \mapsto (\Delta', s)}
\hspace{1.5cm}
  \inferrule*[Right=Ac-Ref]
  { \Delta(r) = s}
  {(\Delta, r) \mapsto (\Delta, s)}
  \\
  %%% Negation rules
  \inferrule*[Right=Ac-Neg]
  { \texttt{negate($\Delta, s$)} = (\Delta', s')}
  { (\Delta, \neg s) \mapsto (\Delta', s')}
\\
  \inferrule*[Right=Ac-C]
  { }
  {(\Delta, \chc[D]{e_{1}, e_{2}}) \mapsto (\Delta, \chc[D]{e_{1}, e_{2}})}
%

  \inferrule*[Right=Ac-Neg-C]
  { }
  { (\Delta, \neg \chc[D]{e_{1},e_{2}}) \mapsto (\Delta, \chc[D]{\neg e_{1}, \neg e_{2}}) }
% \end{mathpar}
\\
% \begin{mathpar}
  \inferrule*[Right=Ac-Unit]
  { }
  { (\Delta, \bullet) \mapsto (\Delta, \unit{})}

%
  \inferrule*[Right=Ac-SOr]
  { \texttt{or($\Delta, s_{1}, s_{2}$)} = (\Delta', s') }
  {(\Delta, s_{1} \vee s_{2}) \mapsto (\Delta', s')}

 \hspace{0.5cm}
  \inferrule*[Right=Ac-SAnd]
  { \texttt{and($\Delta, s_{1}, s_{2}$)} = (\Delta', s') }
  {(\Delta, s_{1} \wedge s_{2}) \mapsto (\Delta', s')}
\\
%
% \begin{mathpar}
%   \hspace{-1.5cm}
  \inferrule*[Right=Ac-DM-Or]
  { }
  { (\Delta, \neg (e_{1} \vee e_{2})) \mapsto (\Delta, \neg e_{1} \wedge \neg e_{2}) }
% \end{mathpar}
%
% \begin{mathpar}
%   \hspace{-1.5cm}
  \\
  \inferrule*[Right=Ac-DM-And]
  { }
  { (\Delta, \neg (e_{1} \wedge e_{2})) \mapsto (\Delta, \neg e_{1} \vee \neg e_{2}) }
  \\
% \end{mathpar}
%
%%% And with Choice rules
% \begin{mathpar}
%   \hspace{-1.5cm}
% \end{mathpar}
%
% \begin{mathpar}
%   \hspace{-1.5cm}
%   \inferrule*[Right=Ac-VAnd-ChcR]
%   {\Delta, e \mapsto \Delta', s}
%   { \Delta, e \wedge \chc[D]{e_{1},e_{2}} \mapsto \Delta', s \wedge \chc[D]{e_{1},e_{2}} }
% \end{mathpar}
%
  %%% Or with Choice rules
% \begin{mathpar}
%   \hspace{-1.5cm}
%   \inferrule*[Right=Ac-VOr-ChcL]
%   {\Delta, e \mapsto \Delta', s}
%   { \Delta, \chc[D]{e_{1},e_{2}} \vee e \mapsto \Delta' \,
%     \chc[D]{e_{1},e_{2}} \vee s}
% \end{mathpar}
%
% \begin{mathpar}
%   \hspace{-1.5cm}
%   \inferrule*[Right=Ac-VOr-ChcR]
%   {\Delta, e \mapsto \Delta', s}
%   { \Delta, e \vee \chc[D]{e_{1},e_{2}} \mapsto \Delta', s \vee \chc[D]{e_{1},e_{2}} }
% \end{mathpar}
%
%%% Congruence rules with symbolic execution
% \end{mathpar}
%
% \begin{mathpar}
%   \hspace{-1.5cm}
%   \inferrule*[Right=Ac-VOr-Value]
%   {(\Delta, e_{1}) \mapsto (\Delta_{1}, s_{1}) \\ (\Delta_{1}, e_{2}) \mapsto
%     (\Delta_{2}, s_{2}) \\  (\Delta_{2}, s_{1} \vee s_{2}) \leadsto
%     (\Delta', s)}
%   { (\Delta, e_{1} \vee e_{2}) \mapsto (\Delta', s)}
% \end{mathpar}
%
  %%% Congruence rules with symbolic execution
  \inferrule*[Right=Ac-VAnd]
  {(\Delta, v_{1}) \mapsto (\Delta_{1}, v_{1}') \\ (\Delta_{1}, v_{2}) \mapsto
    (\Delta', v_{2}')}
  {(\Delta, v_{1} \wedge v_{2}) \mapsto (\Delta', v_{1}' \wedge v_{2}')}

\\

  \inferrule*[Right=Ac-VOr]
  {(\Delta, v_{1}) \mapsto (\Delta_{1}, v_{1}') \\
    (\Delta_{1}, v_{2}) \mapsto (\Delta', v_{2}')}
  {(\Delta, v_{1} \vee v_{2}) \mapsto (\Delta', v_{1}' \vee v_{2}')}
  \\
%
\end{mathpar}
%
% \begin{mathpar}
% \inferrule*[Right=Ac-And]
% {(\Delta, v_{1}) \mapsto (\Delta_{1}, v_{1}') \\ (\Delta_{1}, v_{2}) \mapsto (\Delta_{2}, v_{2}')}
% % ---------------------------------------------------------------
% { (\Delta, v_{1} \wedge v_{2}) \mapsto (\Delta_{2}, v_{1}' \wedge v_{2}') }
% \end{mathpar}
% %
% \begin{mathpar}
% \inferrule*[Right=Ac-Or]
% {(\Delta, v_{1}) \mapsto (\Delta_{1}, v_{1}') \\ (\Delta_{1}, v_{2}) \mapsto (\Delta_{2}, v_{2}')}
% % ---------------------------------------------------------------
% { (\Delta, v_{1} \vee v_{2}) \mapsto (\Delta_{2}, v_{1}' \vee v_{2}') }

% \end{mathpar}

  \caption{Accumulation semantics on IL formulas.}
  \label{impl:accum}
\end{figure}

\begin{figure}
  \begin{mathpar}
  %%% Computation rules
  \inferrule*[Right=Ev-Term]
  { \texttt{assert($(\Gamma, \Delta), t$)} = \Gamma' }
  {((\Gamma, \Delta), t) \rightarrowtail ((\Gamma', \Delta), \unit{}) }
\\
  \inferrule*[Right=Ev-Sym]
  { \texttt{assert($(\Gamma, \Delta), s$)} = \Gamma'}
  { ((\Gamma, \Delta), s) \rightarrowtail ((\Gamma', \Delta), \unit{}) }
\\
  \inferrule*[Right=Ev-Chc]
  { }
  {(\Theta, \chc[D]{e_{1}, e_{2}}) \rightarrowtail (\Theta, \chc[D]{e_{1}, e_{2}})}
%
%   \hspace{0.5cm}
%   \inferrule*[Right=Ev-Unit]
%   { }
%   { (\Theta,\unit{}) \rightarrowtail (\Theta, \unit{}) }
\\
  %%%\unit{} elimination rules
  \inferrule*[Right=Ev-UL]
  {  }
  { (\Theta,\unit{} \wedge v) \rightarrowtail (\Theta, v) }

  \inferrule*[Right=Ev-UR]
  { }
  { (\Theta, v \wedge\unit{}) \rightarrowtail (\Theta, v)}

\\

  \inferrule*[Right=Ev-DM-VOr]
  { }
  { (\Theta, \neg (v_{1} \vee v_{2})) \rightarrowtail (\Theta ,\neg v_{1} \wedge \neg v_{2})}
\\
  \inferrule*[Right=Ev-DM-VAnd]
  { }
  { (\Theta, \neg (v_{1} \wedge v_{2})) \rightarrowtail (\Theta, \neg v_{1} \vee \neg v_{2})}

  %%% Negation rules
  % \inferrule*[Right=Ev-NegS]
  % { (\Delta, \neg s) \mapsto (\Delta', s') \\
  %   ((\Gamma, \Delta'), s') \rightarrowtail (\Theta, \unit{})
  % }
  % { ((\Gamma, \Delta), \neg s) \rightarrowtail (\Theta, \unit{})}
\\
%   \inferrule*[Right=Ev-Neg-Chc]
%   { }
%   { (\Theta, \neg \chc[D]{e_{1},e_{2}}) \rightarrowtail (\Theta, \chc[D]{\neg e_{1}, \neg e_{2}}) }

  \inferrule*[Right=Ev-Neg]
  { (\Delta,\neg v) \mapsto (\Delta', v')
    % ((\Gamma, \Delta') ,\neg v) \rightarrowtail (\Theta, v')
  }
  { ((\Gamma, \Delta), \neg v) \rightarrowtail ((\Gamma, \Delta'), v') }
  \\
  %%% And with Choice rules
% \begin{mathpar}
%   \hspace{-1.5cm}
%   \inferrule*[Right=Ev-And-ChcL]
%   { \Gamma, \Delta,e \rightarrowtail \Gamma', \Delta',e'}
%   { \Gamma, \Delta,\chc[D]{e_{1},e_{2}} \wedge e \rightarrowtail \Gamma',
%     \Delta',\chc[D]{e_{1},e_{2}} \wedge e' }
% \end{mathpar}
% \begin{mathpar}
%   \hspace{-1.5cm}
%   \inferrule*[Right=Ev-And-ChcR]
%   { \Gamma, \Delta,e \rightarrowtail \Gamma', \Delta',e'}
%   { \Gamma, \Delta,e \wedge \chc[D]{e_{1},e_{2}} \rightarrowtail \Gamma',
%     \Delta',e' \wedge \chc[D]{e_{1},e_{2}} }
%   % {l \rightarrowtail l'}
%   % { l \wedge \chc[D]{e_{1},e_{2}} \rightarrowtail l' \wedge \chc[D]{e_{1},e_{2}}}
% \end{mathpar}
%   %%% Or with Choice rules
% \begin{mathpar}
%   \hspace{-1.5cm}
%   \inferrule*[Right=Ev-Or-ChcL]
%   {\Delta,e \mapsto \Delta',v}
%   { \Gamma, \Delta,\chc[D]{e_{1},e_{2}} \vee e \rightarrowtail \Gamma,
%     \Delta',\chc[D]{e_{1},e_{2}} \vee v}
% \end{mathpar}
% \begin{mathpar}
%   \hspace{-1.5cm}
%   \inferrule*[Right=Ev-Or-ChcR]
%   {\Delta,e \mapsto \Delta',v}
%   { \Gamma, \Delta,e \vee \chc[D]{e_{1},e_{2}} \rightarrowtail \Gamma,
%     \Delta',v \vee \chc[D]{e_{1},e_{2}} }
% \end{mathpar}
% \begin{mathpar% }
%   \inferrule*[Right=Ev-Or-Value]
%   {(\Delta, e_{1}) \mapsto (\Delta_{1}, s_{1}) \\
%     (\Delta_{1}, e_{2}) \mapsto (\Delta_{2}, s_{2}) \\
%     (\Delta_{2}, s_{1} \vee s_{2}) \mapsto (\Delta_{s}, s) \\
%     ((\Gamma, \Delta_{s}), s) \twoheadrightarrow (\Theta,\unit{})
%   }
%   { ((\Gamma, \Delta),e_{1} \vee e_{2}) \rightarrowtail (\Theta,\unit{})}
% \end{mathpar}
  %%% Congruence rules with symbolic execution
  \inferrule*[Right=Ev-Or]
  { (\Delta, v_{1}) \mapsto (\Delta_{1}, v_{1}) \\
    (\Delta_{1},v_{2}) \mapsto (\Delta', v_{2})}
  { ((\Gamma, \Delta), v_{1} \vee v_{2}) \rightarrowtail ((\Gamma, \Delta'), v_{1}' \vee v_{2}')}

  \inferrule*[Right=Ev-And]
  {(\Theta, v_{1}) \rightarrowtail (\Theta_{1}, v_{1}') \\
    (\Theta_{1}, v_{2}) \rightarrowtail (\Theta', v_{2}')
  }
  { (\Theta, v_{1} \wedge v_{2}) \rightarrowtail (\Theta', v_{1}' \wedge v_{2}')
  }
\end{mathpar}%

  \caption{Evaluation semantics on IL formulas.}
  \label{impl:accum}
\end{figure}

\begin{figure}
  % \begin{tabbing}
%   % \begin{align*}
%   {\sc CoreChoices}$\ :\ C \rightarrow vm \rightarrow ev \rightarrow vm$ \\
%   {\sc CoreChoices}$\ C\ mdls\ v$\\
%   \qquad case $v$ of \\
%   \qquad \qquad \= Unit \qquad \qquad \= = return \mdls{} \\
%   \> \chc[d]{e,e'} \> = CoreChoicesHelper C \dimd{} \mdls{} (Evaluate e) (Evaluate e') \\
%   \> \chc[d]{e,e'} $\wedge\ ev\ $ \> = do \\
%   \> \> \quad \=  $vE \leftarrow$ Evaluate $e$ \\
%   \> \> \> $vE' \leftarrow$ Evaluate $e'$ \\
%   \> \> \> CoreChoicesHelper C \dimd{} \mdls{} ($vE \wedge\ ev$) ($vE' \wedge ev$) \\
%   \> $ev\ \wedge\ $\chc[d]{e,e'} \> = do \\
%   \> \> \quad \=  $vE \leftarrow$ Evaluate $e$ \\
%   \> \> \> $vE' \leftarrow$ Evaluate $e'$ \\
%   \> \> \> CoreChoicesHelper C \dimd{} \mdls{} ($ev \wedge\ vE$) ($ev \wedge eV'$) \\
%   \> \chc[d]{e,e'} $\vee\ ev\ $ \> = do \\
%   \> \> \quad \=  $vE \leftarrow$ Evaluate $e$ \\
%   \> \> \> $vE' \leftarrow$ Evaluate $e'$ \\
%   \> \> \> CoreChoicesHelper C \dimd{} \mdls{} ($vE \vee\ ev$) ($vE' \vee ev$) \\
%   \> $ev\ \vee\ $\chc[d]{e,e'} \> = do \\
%   \> \> \quad \=  $vE \leftarrow$ Evaluate $e$ \\
%   \> \> \> $vE' \leftarrow$ Evaluate $e'$ \\
%   \> \> \> CoreChoicesHelper C \dimd{} \mdls{} ($ev \vee\ vE$) ($ev \vee eV'$) \\
%   \> $ev$ \> = do \\
%   \> \> \quad \= $vE \leftarrow$ FindPChoice (Evaluate $ev$) \\
%   \> \> \> solveChoices C \dimd{} \mdls{} ($vE$)  \\
% \end{tabbing}
\begin{mathpar}
  %%% Computation rules
  \inferrule*[Right=Gen-M]
  { \texttt{genModel($\Phi$)} = m }
  { (\Phi, \unit{}) \Downarrow_{i} m }

  \inferrule*[Right=Cr-Sym]
  { ((\Gamma, \Delta), s) \rightarrowtail ((\Gamma', \Delta'), \unit{})}
  { ((C, \Gamma, \Delta), s) \Downarrow_{i} ((C, \Gamma', \Delta'), \unit{})
  }
\end{mathpar}
%
\begin{mathpar}
\inferrule*[Right=Cr-And]
{ (\Phi, v_{1}) \Downarrow_{i} (\Phi_{1}, v_{1}') \\
    (\Phi_1, v_{2}) \Downarrow_{1} (\Phi', v_{2}') \\
  }
  { (\Phi, v_{1} \wedge v_{2}) \Downarrow_{i} (\Phi', v_{1}' \wedge v_{2}')
  }
\\%
%   \hspace{0.5cm}
  %% solve a symbolic reference execute
  \inferrule*[Right=Cr-Or]
  { (\Delta, v_{1} \vee v_{2}) \mapsto (\Delta', v)}
  { ((C, \Gamma, \Delta), v_{1} \vee v_{2}) \Downarrow_{i} ((C, \Gamma, \Delta'), v) }

\\%
%%%%%%% And rules %%%%%%%%%%%%%%%%%%%%%
\inferrule*[Right=Cr-And-TR]
{
  C(D) = \true \\
  ((C, \Gamma, \Delta), v \wedge \texttt{toIR$(e_{1})$}) \Downarrow_{i} m
}
{
  ((C, \Gamma, \Delta), v \wedge \chc[D]{e_{1}, e_{2}}) \Downarrow_{i} m
}
\\
\inferrule*[Right=Cr-And-TL]
{
  C(D) = \true \\
  ((C, \Gamma, \Delta), \texttt{toIR$(e_{1})$} \wedge v) \Downarrow_{i} m
}
{
  ((C, \Gamma, \Delta), \chc[D]{e_{1}, e_{2}} \wedge v) \Downarrow_{i} m
}

\\%
\inferrule*[Right=Cr-And-FR]
{
  C(D) = \false \\
  ((C, \Gamma, \Delta), v \wedge \texttt{toIR$(e_{2})$}) \Downarrow_{i} m
}
{
  ((C, \Gamma, \Delta), v \wedge \chc[D]{e_{1}, e_{2}}) \Downarrow_{i} m
}
\\
\inferrule*[Right=Cr-And-FL]
{
  C(D) = \false \\
  ((C, \Gamma, \Delta), \texttt{toIR$(e_{2})$} \wedge v) \Downarrow_{i} m
}
{
  ((C, \Gamma, \Delta), \chc[D]{e_{1}, e_{2}} \wedge v) \Downarrow_{i} m
}

%%%%%%%%%%%%%%%%%%%%%%%% Or rules %%%%%%%%%%%%%%%%%%%%%%%%%%%%%%%%%%%

\inferrule*[Right=Cr-Or-TR]
{
  C(D) = \true \\
  ((C, \Gamma, \Delta), v \vee \texttt{toIR$(e_{1})$}) \Downarrow_{i} m
}
{
  ((C, \Gamma, \Delta), v \vee \chc[D]{e_{1}, e_{2}}) \Downarrow_{i} m
}
\\
\inferrule*[Right=Cr-Or-TL]
{
  C(D) = \true \\
  ((C, \Gamma, \Delta), \texttt{toIR$(e_{1})$} \vee v) \Downarrow_{i} m
}
{
  ((C, \Gamma, \Delta), \chc[D]{e_{1}, e_{2}} \vee v) \Downarrow_{i} m
}

\\%
\inferrule*[Right=Cr-Or-FR]
{
  C(D) = \false \\
  ((C, \Gamma, \Delta), v \vee \texttt{toIR$(e_{2})$}) \Downarrow_{i} m
}
{
  ((C, \Gamma, \Delta), v \vee \chc[D]{e_{1}, e_{2}}) \Downarrow_{i} m
}
\\
\inferrule*[Right=Cr-Or-FL]
{
  C(D) = \false \\
  ((C, \Gamma, \Delta), \texttt{toIR$(e_{2})$} \vee v) \Downarrow_{i} m
}
{
  ((C, \Gamma, \Delta), \chc[D]{e_{1}, e_{2}} \vee v) \Downarrow_{i} m
}


%%%%%%%%%%%%%%%%%%%%%%%%%%%%%%%% Choice rules %%%%%%%%%%%%%%%%%%%%%%%%%%%%%%%%%%
  \inferrule*[Right=Cr-CAnd-R]
  {
    \hspace{1.3cm}
    ((C \cup \{(D, \true)\}, \Gamma, \Delta), v \wedge \chc[D]{e_{1}, e_{2}}) \Downarrow_{i+1} m_{1} \\
    D \notin C \\
    ((C \cup \{(D, \false)\}, \Gamma, \Delta), v \wedge \chc[D]{e_{1}, e_{2}}) \Downarrow_{i+1} m_{2} \\
  }
  {
    ((C, \Gamma, \Delta), v \wedge \chc[D]{e_{1}, e_{2}}) \Downarrow_{i} m_{1}
    \oplus m_{2}
  }
  \\
  \inferrule*[Right=Cr-COr-R]
  {
    \hspace{1.3cm}
    ((C \cup \{(D, \true)\}, \Gamma, \Delta), v \vee \chc[D]{e_{1}, e_{2}}) \Downarrow_{i+1} m_{1} \\
    D \notin C \\
    ((C \cup \{(D, \false)\}, \Gamma, \Delta), v \vee \chc[D]{e_{1}, e_{2}}) \Downarrow_{i+1} m_{2} \\
  }
  {
    ((C, \Gamma, \Delta), v \vee \chc[D]{e_{1}, e_{2}}) \Downarrow_{i} m_{1}
    \oplus m_{2}
  }
  \\

  \inferrule*[Right=Cr-CAnd-L]
  {
    \hspace{1.3cm}
    ((C \cup \{(D, \true)\}, \Gamma, \Delta), \chc[D]{e_{1}, e_{2}} \wedge v) \Downarrow_{i+1} m_{1} \\
    D \notin C \\
    ((C \cup \{(D, \false)\}, \Gamma, \Delta), \chc[D]{e_{1}, e_{2}} \wedge v) \Downarrow_{i+1} m_{2} \\
  }
  {
    ((C, \Gamma, \Delta), \chc[D]{e_{1}, e_{2}} \wedge v) \Downarrow_{i} m_{1}
    \oplus m_{2}
  }
  \\
  \inferrule*[Right=Cr-COr-L]
  {
    \hspace{1.3cm}
    ((C \cup \{(D, \true)\}, \Gamma, \Delta), \chc[D]{e_{1}, e_{2}} \vee v) \Downarrow_{i+1} m_{1} \\
    D \notin C \\
    ((C \cup \{(D, \false)\}, \Gamma, \Delta), \chc[D]{e_{1}, e_{2}} \vee v) \Downarrow_{i+1} m_{2} \\
  }
  {
    ((C, \Gamma, \Delta), \chc[D]{e_{1}, e_{2}} \vee v) \Downarrow_{i} m_{1}
    \oplus m_{2}
  }
%
\end{mathpar}

  \caption{Variational solving semantics on cores.}
  \label{impl:choice-eval}
\end{figure}

\end{document}
\endinput
